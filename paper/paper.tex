\documentclass[12pt,a4paper]{article}
\usepackage{amsfonts}
\usepackage{algorithm2e}
\usepackage{amsthm}
\usepackage{amsmath}
\usepackage{amscd}
\usepackage{booktabs}
\usepackage[latin2]{inputenc}
\usepackage{t1enc}
\usepackage[mathscr]{eucal}
\usepackage{indentfirst}
\usepackage{graphicx}
\usepackage{hyperref}
\usepackage{graphics}
\numberwithin{equation}{section}
\usepackage{hyperref}
\usepackage[margin=2.9cm]{geometry}
\usepackage[tableposition = top]{caption}
\usepackage[outdir=./images/]{epstopdf}
\epstopdfsetup{outdir=./images/}

\def\numset#1{{\\mathbb #1}}

\begin{document}
 
\title{Textplainer: Text Feature Explanations for Arbitrary Machine Learning Models}

\author{John Hawkins} 

\maketitle

\section{Abstract}

Explaining what contributes to the output of machine learning models is of increasing importance in
a world where machine learning is embedded in critical decision making processes. In spite of recent advances in
the explainability of model decisions, complex structured data like raw text remains difficult to explain.

In this work we present a system for explaining some of the contributions of text to machine learning models
in a manner that is sensitive to some of the specific qualities of text. We lay the ground work for additional
work to expand this technique and provide context specific explanation of the way text is utilised in a machine
learning system.

We have made the code available in an Open Source GitHub repository 
available at \href{https://github.com/john-hawkins/textplainer}{https://github.com/john-hawkins/textplainer}
and released using the PyPi library. 


\section{Introduction}

Explaining how each individual data point or feature contributes to the output of a machine learning model is 
critical to both the analysis of the model during its development, the scutiny of the model during review and 
often to its ongoing use in production. The contributions of features are typically analysed at two different
scales, global feature importance and local feature importance. The global importance simply tells us over a large 
set of predictions the extent to which an individual feature is contributing to predictions. It is a measure of overall
importance, but typically tells us very little about individual predictions.

By contrast, local feature importance refers to a range of techniques that seek to explain the contribution of a given
value of a specific variable to the output of a model on a case by case basis. In a linear model this is a trivial task
because the answer can be gleaned by simply inspecting the coefficients of the model. For complex non-linear models
this is considerably more difficult. In recent years techniques like Lime\cite{lime}, Xemp\cite{xemp} and Shap\cite{shap}
have made great in-roads into
delivering similar kinds of insights into non-linear machine learning models but none of them make satisfactory progress
with raw text. The reason being that they all require that we are able to reference the contribution of the variable to
its mean value, local mean value or some sample of other values. Text data, because of its extremely high dimensionality,
does not work well with these techniques. They can be applied to the processed data, i.e. to the features extracted from
the text but this does not improve the quality of the explanations because they are extremely high dimensional vector
spaces far-removed from human comprehension.

There has been considerable work done to try and build models with improved explainability for text variables. 
Arras et al XXX design an algorithm that XXX \cite{Arras2017}. In all of these cases the authors are designing a specific machine learning algorithm
with improved explainability features, rather than explaining the performance of existing models.

Part of the key difficulty of explaining how text contributes to a model is that there are an open ended number of ways in
which the text can vary. There are multiple conceptual levels at which a human might analyse the features of the text
to understand the relationship between the tokens, the meaning and  the intention. By way of example, all of the following
are potential explanations of why a text might be "difficult to understand" which is a potential real-world application of a
text classifier.

\begin{itemize}
   \item "There are too many weird and unusual words in this article."
   \item "The sentences are all very long and structurally complex."
   \item "There are so many spelling and grammatical mistakes I cannot concentrate on understanding it."
\end{itemize}

This list is by no means exhaustive and should only hint at the complexity of the problem. In order to make progress with this task we
present a technqiue for analysing the component contributions of the words in a text across four key dimensions: length, existance,
specificty and polarity (LESP). The result is a human readable illustration of how text contributes to the predictions made by a model
across these four dimensions, which should assist the user in deriving higher level explanations. 


\section{Method}

The essence of many of the local feature importance techniques is to illustrate how much the prediction of the model would have been
different if the value of the specific variable had been missing or given an average value. This difference from some kind of neutral
value is more conceptually difficult with text. Replacing the text with a null value will give you a sense of the overall contribution
of the text (presuming the model can cope with this), however, it does nothing to explain what it is about the text that is making the
observed contribution relative to the null.

We start our analysis by calculating the diffence between the record scored with the true value and the record scored with the value replaced
with an empty string. The choice of the empty string is a deliberate choice to avoid the complication that NULL records are sometimes used
by systems to indicate the absence of particular actions in the system that recorded them. The goal of the technique is to explain the individual
contributions inside the text to the overall text.

Due to the fact that the potential number of contributing structures in the text is computationally prohibitive, our first goal is to identify
those tokens in the text that contribute the most through their existance. We do so by re-scoring the record with all each of the word tokens
removed individually. This has a computational complexity capped at the length of the text (by default we exclude stop-words), and we can
achieve additional computational gains by pre-generating these records and scoring them all in parrallel. The contribution of token's
existance ($E$ for word $W_i$ is calculated as shown in Equation \ref{eq:E}

\begin{equation}
\label{eq:E}
E(W_i) = \frac{ f(\mathscr{T}) - f( d(\mathscr{T},i) ) }{ f(\mathscr{T}) - f(\epsilon) }
\end{equation}

Where our arbitray machine learning model is demarcated by the function $f()$, which for our purposes will operate on a text block $\mathscr{T}$.
We ignore all other variables passed to the model as they remain constant throughout this analysis. The function $d$ will delete the $i^{th}$ word
token from the text sequence, and $\epsilon$ is the empty string.

We then sort the values of $E$ for the individual word tokens and take the top $N$. These word tokens form the basis for all additional analysis
and limit the computational search space. Alternatively, this can be set to take all word tokens that contribute above a threshold proportion of the
overall text contribution.

In order to calculate the remaining scores for the key word tokens we need to match them against a custom dictionary of synonyms and antonyms. This
dictionary has been derived from the project gutenberg copy of Samuel Fallows dictionary of synonymns \cite{fallows}. We have processed this dictionary
into a data structure such that we are able to retrieve a list of synonyms and antonyms for each word token in the list. In addition we have added entries for a range of common mispellings. We then generate new entries for the record in which the word token is replaced by all of the synonyms and antonyms. These
are scored in parrallel as per the existance round. The scores for $I$ and $P$ are calculated as show in Equations \ref{eq:I} and \ref{eq:P} respectively.


\begin{equation}
\label{eq:I}
I(W_i) = \frac{ f(\mathscr{T}) - \frac{ \sum_{s=1}^{|\mathcal{S}|} f( r(\mathscr{T},i,\mathcal{S}_s) )}{|\mathcal{S}|} }{ f(\mathscr{T}) - f( d(\mathscr{T},i) ) }
\end{equation}
  
\begin{equation}
\label{eq:P}
P(W_i) = \frac{ f(\mathscr{T}) - \frac{ \sum_{a=1}^{|\mathcal{A}|} f( r(\mathscr{T},i,\mathcal{A}_a) )}{|\mathcal{A}|} }{ f(\mathscr{T}) - f( d(\mathscr{T},i) ) }
\end{equation}

Where $\mathcal{S}$ and $\mathcal{A}$ are the sets of synonyms and antonyms for word token $i$ respectively, and the function $r(\mathscr{T},i,w)$ will 
replace the $i^{th}$ word token with $w$. 
These metrics represent the difference between the contribution of the word found in the text and the average impact of its synonyms or antonyms
on the output of the model.  In both cases these metrics are calculated relative to the numerator of the \textit{existance} metric so that
their contribution is relative to the presence of the word, rather than the presence of the entire text.


\section{Experiments}

To demonstrate the textplain technique we have built models on two public domain datasets using multiple different machine learning techniques.
%The datasets are the Amazon Fine Food reviews by McAuley and Leskovec \cite{McAuley2013},
The datasets are the Sentiment140 Twitter sentiment data and by Go, Bhayani and
Huang \cite{Go2009} and the dataset of SMS Spam by Almeida and Gómez Hidalgo \cite{Almeida2011}. 
In the first problem we are predicting the sentiment of a tweet from its text content where the label was derived from an emoticon in the 
original tweet. In effect this amounts to predicting the sentiment of the emoticon that will be added to a tweet from its text.
In the second case we are predicting which SMS messages are spam from the message text itself.

To illustrate the effectiveness of textplain on diverse machine learning models we apply the following three
machine learning pipelines to each dataset.

\begin{enumerate}
   \item A bi-gram encoding of the text with TFIDF encoding used to train a regularized logistic regression.
   \item A neural network with a learned embedding trained directly from a vector of word tokens 
   \item A BERT derived classifier.
\end{enumerate}
  



\section{Results}


\section{Discussion}


\bibliographystyle{splncs}
\bibliography{refs}

\end{document}


